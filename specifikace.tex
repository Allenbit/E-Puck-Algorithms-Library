\documentclass[11pt]{article}

\usepackage{a4wide}
\usepackage[czech]{babel} 
\usepackage[utf8]{inputenc}
\usepackage[T1]{fontenc}
\usepackage{graphicx}
\usepackage{charter}
\usepackage{float}
\usepackage{indentfirst}

%\newcommand\uv[1]{\quotedblbase #1\textquotedblleft}%

\floatstyle{ruled}
\newfloat{vystup}{H}{lop}
\floatname{vystup}{Výstup}

\linespread{1.3}

\begin{document}
\author{David Marek}
\title{Specifikace ročníkového projektu\\Knihovna algoritmů pro E-Puck robota}
\date{}
\maketitle{}

\section{Záměr ročníkového projektu}

Cílem tohoto ročníkového projektu je vytvořit jednak prostředek pro komunikaci
s E-Puck robotem a jeho ovládání z klasického počítače pomocí bluetooth, ale
také vytvořit přehled algoritmů, na kterých bude ukázáno jaké problémy dokáže
E-Puck robot řešit a jak nejlépe postupovat pro jejich zdolání. Výsledek této
práce by měl sloužit převážně k výukovým účelům, jako referenční materiál pro
studenty, kteří budou chtít pracovat s E-Puck robotem. Algoritmy vytvořené v
rámci této práce by měli být schopni lehce začlenit do svých programů, stejně
tak jako by se měli být schopni ze zdrojového kódu poučit o nejlepších cestách
při práci na svých problémech.

Hlavní důraz tedy bude kladen na využitelnost jednotlivých algoritmů. Cílem
není vytvořit dokonalý program pro robota. Naopak, cílem je vytvořit masivní
základní kameny, které bude možné využít mnohokrát, a které usnadní práci s
E-Puck robotem. Dále bude kladen důraz na přehlednost zdrojového kódu, jeho
zdokumentovanost, která by měla přerůst standardní zdrojové kódy, protože tady
se bude očekávat, že algoritmy budou sloužit jako inspirace a měly by tedy být
velmi snadno pochopitelné.

\section{E-Puck robot}

Jedná se o malého robota vyvinutého pro výukové účely a výzkum. Jeho hlavní
předností jsou rozměry, ty umožňují pracovat s robotem na stole vedle počítače
a hned si zkoušet napsané programy. Další z výhod E-Puck robota je, že obsahuje
mnoho senzorů, díky kterým je možné zkoušet mnoho různých algoritmů.

\section{Použité technologie}

Komunikace mezi počítačem a E-Puck robotem bude probíhat přes bluetooth. V
E-Puck robotovi bude nahrán firmware napsaný v jazyku C. Nepůjde o vlastní
invenci, nýbrž o firmware BTCom přímo od výrobců E-Puck robota, ten komunikuje s
počítačem pomocí jednoduchého textového protokolu. Knihovna pro ovládání a
komunikaci s robotem bude napsána v jazyku Python. Knihovna samotná by měla být
multiplatformní, avšak doplňující utility (např. pro připojení robota) budou
pro OS Linux. Pro jiné operační systémy existují alternativy, avšak těmi se
nebudu zabývat. Rozměry robota, bluetooth připojení a python vytváří spojení,
díky kterému bude možné ovládat robota v reálném čase a testovat na něm
algoritmy za letu. Další výhodou je, že python je velmi lehké použít pro
skriptování z jiných programovacích jazyků (např. C++), tedy použití nebude
omezeno pouze pro programátory v pythonu. Knihovna bude zprostředkovávat
následující vlastnosti robota:

\begin{itemize}
\item Krokové motory -- je možné nastavit rychlost pro každý motor zvlášť, získat
    infomace o počtu vykonaných kroků a z toho odvodit vzdálenost, nebo naopak
    odměřit přesný počet kroků a tak na vhodném povrchu robota přesně navádět.
\item Senzory překážek -- robot obsahuje 8 sensorů překážek dokola po celém těle.
    Samozřejmě, že většina jich je situována do přední části robota, tedy je
    možné získat dostatečný počet dat pro detekci překážek.
\item Světelné senzory -- infračervené senzory pro detekci překážek lze přepnout
    do pasivního módu, kdy slouží k detekci světla.
\item Akcelerometr -- robot umožňuje zjišťovat sílu na něj působící jako 3D
    vektor.
\item Mikrofony -- na horní straně robota jsou 3 mikrofony umístěné tak, že je
    lze použít k triangulaci zdroje zvuku.
\item Kamera -- Ve předu má robot malou kamerku, kterou je možné použít např. pro
    detekci tvarů, anebo jako lineární kameru pro sledování cesty. 
\end{itemize}

Toto jsou všechny technologie, které robot obsahuje, a pro které by měly být
napsány algoritmy.

\section{Existující práce}

\subsection{Cyberbotics' Robot Curriculum}

Jedná se o knížku od Oliviera Michela, dále prohloubenou pracemi v EPFL od
Fabiena Rohrera a Nicolase Heinigera. Je šířena pod svobodnou licencí a volně
dostupná na wikibooks na adrese
http://en.wikibooks.org/wiki/Cyberbotics'\_Robot\_Curriculum. Tato knížka se
zabývá E-Puck roboty, používá programy napsané v jazyku C a testované v
simulátoru Webots. Čitatel je veden pomocí řešení úkolů, od jednoduchého
pohybování, přes používání IR senzorů, akcelerometru, kamery, až k umělé
inteligenci založené na chování a později i na neuronových sítích. Vždy je před
něj předložen problém vymodelovaný v prostředí Webots a jeho úkolem je napsat
program, který jej řeší.

V čem se bude práce lišit od této knížky? V této práci budu předpokládat, že
uživatel mé knihovny ví co chce vytvořit, ale nechce se zabývat přílišnými
detaily, popř. se chce inspirovat jak je rešit elegantně. Také půjde v
některých případech o algoritmy, na které by robot kapacitně nestačil, tedy
kontrolní program bude běžet na počítači a do robota se budou přenášet pouze
příkazy. Výhodou knížky je použití emulátoru Webots, kde lze simulovat fyziku
robota. Avšak E-Puck robota je s jeho rozměry možné bez problémů testovat na
stole u počítače. Není třeba pak řešit odchylky simulace od reality.

\subsection{Learning Computing With Robots}

Knížka vyučuje programování v pythonu na příkladu ovládání robota Scribbler.
Robot je zde dost často pouze jako lákadlo, v některých kapitolách se jím autor
ani nezabývá. Zajímavé jsou kapitoly o hmyzím chování (Braitenberg's vehicles)
a rozeznávání obrazu. Ve své práci bych chtěl prozkoumat stejná témata, ale jít
do větší hloubky, například v knížce je u rozeznávání obrazu pouze metoda
hledání objektu podle filtrování barvy.

\subsection{Webots}

Jedná se o vývojové prostředí pro vývoj robotů. Obsahuje také 3D simulátor se
simulací fyziky. Obsahuje základní knihovnu pro ovládání robota. Ta je
použitelná na více druhů robotů (např. E-Puck, Khepera, \ldots). Program může
být simulován na počítači a pak nahrán do robota. V mé práci se ale chci
zabývat i algoritmy, které by na robotovi nemohly reálně běžet z důvodu jeho
slabého výkonu. Prostředí Webots umožňuje ovládání robota z počítače pouze v
placené verzi.

\subsection{Evorobot*}
Jedná se o program pro simulaci experimentů na evoluci koletivního chování.
Příklady jsou z knihy Evolutionary Robotics. Ta se zabývá pouze evolučními
algoritmy a jejich možnostmi pro vývoj programu robota. Ve své práci se
nehodlám evolučními algoritmy příliš zabývat, možná jen okrajově. Hlavním
důvodem je nutnost dlouhého provádění experimentů pro učení robota. S největší
pravděpodobností by bylo potřeba použít simulátoru a i tam pak můžou nastat
problémy s adaptací na reálného robota.

\subsection{Pyro}
Python Robotics je projekt zabývající se zkoumáním témat z umělé inteligence a
robotiky, snaží se odprostit od nízkoúrovňových věcí a tedy i od závislosti na
specifické architektuře. Funguje na různých robotech (E-Puck mezi nimi ale
není). Ve své práci se nechci povznést až do tak abstraktních výšin. Stejně tak
se nebudu výhradně zabývat učením, evolučními algoritmy a podobnými tématy, na
které je Pyro zaměřen.

\section{Rozdělení práce}
\subsection{Komunikace s robotem}
Robot bude ovládán přes bluetooth spojení. Robot vytváří s počítačem sériové
spojení. Posílání příkazů je tedy velmi jednoduché. V pythonu bude použita
knihovna pyserial, ale je možné s robotem komunikovat i pomocí sériových
terminálů, jako je třeba cutecom. Na straně počítače bude komunikovat nově
napsaná nízkoúrovňová knihovna, na straně robota bude použita knihovna BTCom,
která patří do standardní knihovny E-Puck robota. Toto spojení bude umožňovat z
počítače ovládat celého robota stejně tak dobře, jako by to šlo z firmwaru
nahraného v něm.

V této části bude definován objekt Controller, který bude mít velmi přimočaré metody:
\begin{itemize}
\item Nastavení/čtení rychlosti kol, nastavení/čtení pozice motoru.
\item Ovládání LED na těle robota, přední LED, LED v těle robota.
\item Získávání informací z akcelerometru ve formě vektoru.
\item Čtení hodnoty 16 polohového kolečka na robotovi.
\item Nastavení parametrů kamery, získání obrázku.
\item Získání informací z IR senzorů (vzdálenost, světlo).
\item Získání informací z mikrofonů, přehrání zvuku ze speakeru.
\end{itemize}

\subsection{Pohyb}

V této části budou algoritmy pro pohyb robota:
\begin{itemize}
\item Kalibrace robota vůči povrchu (na špatném povrchu robot může na jeden krok
motoru ujet jinou vzdálenost, než by se od něj očekávalo).
\item Ujetí zadané vzdálenosti.
\item Otočení se o zadaný úhel.
\end{itemize}
A dále budou následovat příklady programů pro pohyb:
\begin{itemize}
\item Jednoduché pravidla vytvářející komplexní chování (Braitenberg vehicles).
\item Vyhýbání se překážkám.
\item Pohyb v bludišti.
\end{itemize}

\subsection{Detekce překážek}
Použití IR senzorů pro vyhýbání se překážkám, využití akcelerometru pro detekci
překážek nezjistitelných IR senzory. Anebo pro reakci na kolizi s objektem.
Detekce náklonu a využití například pro balancování.

\subsection{Využití kamery}
E-Puck robot má kameru, která dokáže snímat v rozlišení 640x480 pixelů.
Problémem ovšem je, že uvnitř robota není dostatečně velký buffer pro uložení
obrázku a jeho následné zpracování. Standardně se to řeší tak, že kamera
umožňuje snímat jen každý x-tý pixel. Ale díky tomu, že v mé práci budu psát
algoritmy, které poběží na počítači, tak si můžu dovolit například složit více
obrázku do jednoho většího a s ním pak pracovat.

Nejčastější využití kamery jsou dvě:
\begin{itemize}
\item Detekce objektů -- plánuji do své práce zahrnout algoritmy a
transformace, které by měly detekci objektů a samotnou práci se získanými
obrázky zjednodušit. Mělo by se jednat o algoritmy různé složitosti, od
jednoduchého hledání objektu určité barvy, přes detekci hran, rozeznávání
obrazců, až po složitější algoritmy z počítačového vidění. Dále bude práce
obsahovat ukázky využití těchto algoritmů, např. robot se bude snažit dostat k
specifickému obrazci, anebo bude sledovat pohybující se objekty.

\item Lineární kamera -- kameru je možné přepnout do režimu, kdy fotí v
rozlišení 640x1 pixelů. Taková fotka se pak dá např. použít pro sledování čáry
na zemi. Hodlám ukázat nějaké příklady, například lze použít barevnou fotku, na
zem nakreslit více čar různými barvami a pomocí filtrování barev se robot může
rozhodovat, kterou čáru bude sledovat.
\end{itemize}

\subsection{Práce se zvukem}
E-Puck obsahuje 3 mikrofony, jsou rozmístěné do trojúhelníku na horní straně
robota. Díky tomu se dají využít třeba k triangulaci. V práci určitě nezapomenu
na algoritmy pro zpracování zvuku. Příkladem může být robot, který bude
přijímat příkazy jako tóny (krysař), anebo bude jen následovat zdroj zvuku.

Další periferií, kterou E-Puck obsahuje, je speaker. Využitím může být
oznamování událostí. V tomto ohledu budu muset provést hlubší zkoumání,
vzhledem k tomu, že podle knihovny BTCom umí robot přehrát jen 5 tónu, ale
domnívám se, že by měla být schopná úprava tak, aby mohl vydávat libovolný
zvuk. Podle některých zdrojů by totiž mělo jít i přehrát WAV soubor.

\subsection{Komunikace mezi roboty}
Jedná se o téma, které bych rád prozkoumal, ale nejsem si jistý, zda-li budu
mít přístup k více robotům současně.

Roboti disponují velkou řádkou možností, jak se mezi sebou domlouvat. Mohou se
označovat rozsvícením LED diod a pak hledat kamerou. Další možností je využít
jejich IR senzory k přenosu dat na krátkou vzdálenost, zde je už ale potřeba
úprava firmware robota. Další možností je používat mikrofony a speaker.
Komunikace by pak probíhala pomocí zvuku. Ten se dá využít nejen ke komunikaci,
ale také k synchronizaci. Pokud má více robotů provádět stejný program a mají
začít ve stejný čas, tak nejlepší způsob jak je synchronizovat, je nechat je
čekat do zvukového signálu (tlesknutí, prasknutí balónu, \ldots). 

\subsection{Evoluční robotika}
Je možné robota naučit požadovanému chování pomocí hodnocení jeho výkonu
fitness funkcí a aplikací učení a evoluce na jeho ovládací program vytvořený
neuronovou sítí. Publikací, která se touto problematikou zabývá je Evolution
Robotics (napsal Stefano Nolfi a Dario Floreano). V jejich knize je spousta
příkladů, které využívají učení robota. Rád bych si tento přístup vyzkoušel a
vytvořil nějaký příklad, avšak s jedním robotem může být vyzkoušení
dostatečného počtu generací práce na dlouhý běh.

\section{Závěr}
Specifikoval jsem, kterými tématy se chci zaobírat. Procentuální zastoupení
jednotlivých témat ve výsledné práci bude záležet na možnostech a pracovních
podmínkách, vzhledem k tomu, že některé pokusy mohou vyžadovat více hardwaru
nebo příliš mnoho času.

\end{document}
